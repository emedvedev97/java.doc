\documentclass[a4paper]{article}
\usepackage[14pt]{extsizes} % для того чтобы задать нестандартный 14-ый размер шрифта
\usepackage[utf8]{inputenc}
\usepackage[russian]{babel}
\usepackage{setspace,amsmath}
\usepackage[left=20mm, top=15mm, right=15mm, bottom=15mm, nohead, footskip=10mm]{geometry} % настройки полей документа
\usepackage{graphicx}


\begin{document} % начало документа
% НАЧАЛО ТИТУЛЬНОГО ЛИСТА
\begin{center}
\hfill \break
\normalsize{\textbf{Федеральное государственное бюджетное образовательное}}\\ 
\normalsize{\textbf{учреждение высшего образования}}\\
\small{\textbf{«РЫБИНСКИЙ ГОСУДАРСТВЕННЫЙ АВИАЦИОННЫЙ ТЕХНИЧЕСКИЙ}}\\
\small{\textbf{УНИВЕРСИТЕТ имени П. А. СОЛОВЬЁВА»}}\\
\hfill \break
\normalsize{Факультет радиоэлектронники и информатики}\\
 \hfill \break
\normalsize{Кафедра математического и программного обеспечения электронных вычислительных средств}\\
\hfill\break
\hfill \break
\hfill \break
\hfill \break
\large{КУРСОВАЯ РАБОТА}\\
\normalsize{\textbf{«по дисциплине «Промышленная разработка программ на основе java-технологий»}}\\
\hfill \break
\normalsize{{на тему}}\\


\normalsize{Разработка WEB-приложения «Стоматологическая клиника»
}\\
\normalsize{Пояснительная записка
}\\
\hfill \break
\hfill \break
\end{center}
\hfill \break
\begin{flushright}
{Выполнили:} \\
{Паламарь А.И.} \\
{Медведев Е.Ю.} \\
{Проверил:} \\
{Волков М.Л.} \\
\end{flushright}
\hfill \break
\hfill \break
\hfill \break
\hfill \break
\hfill \break
\hfill \break
\begin{center} Рыбинск 2020 \end{center}
\thispagestyle{empty} % выключаем отображение номера для этой страницы

\newpage

\hfill \break
\hfill \break
\begin{center}
\Large{\textbf{Содержание}}\\
\end{center}
\normalsize{Введение................................................................................................................3}\\
\normalsize{1.  Функциональные требования...........................................................................4}\\
\normalsize{2. Программное обеспечение................................................................................6}\\
\normalsize{3. Разработка базы данных	.................................................................................7}\\
\normalsize{4. Разработка web-сервиса	...................................................................................9}\\
\normalsize{5. Разработка клиентского приложения............................................................12}\\
\normalsize{Заключение..........................................................................................................13}\\
\normalsize{Список использованных источников..................................................................14}\\


\newpage

\hfill \break
\hfill \break
\begin{center}
\Large{\textbf{Введение}}\\
\end{center}


\normalsize{WEB-служба, WEB -сервис — это сетевая технология, обеспечивающая межпрограммное взаимодействие на основе веб-стандартов. Консорциум W3C определяет веб-сервис, как «программную систему, разработанную для поддержки интероперабельного межкомпьютерного (machine-to-machine) взаимодействия через сеть».}\\


\normalsize{В данной курсовой работе необходимо разработать web-сервис для стоматологической клиники с возможностью записи пациента на прием, просмотром им своей истории болезни, с точки зрения врача просматривать страницу текущего приема, с точки зрения администратора или модератора – добавлять услуги клиники и распределять услуги между специалистами.
}\\


\normalsize{Целью данной курсовой работы является является закрепление знаний, полученных в течение курса «Промышленная разработка программ на основе java-технологий» и написание web-приложения «Стоматологическая клиника».}\\


\normalsize{Для достижения полученной цели были поставлены следующие задачи:}
\begin{itemize}
\item разработка и создание базы данных;
\item написание web-сервиса;
\item написание клиентского приложения.
\end{itemize}\\

\newpage

\hfill \break
\hfill \break
\begin{center}
\Large{\textbf{1.  Функциональные требования}}\\
\end{center}


\normalsize{В данной курсовой работе необходимо разработать web-сервис для стоматологической клиники с возможностью записи пациента на прием, просмотром им своей истории болезни, с точки зрения врача просматривать страницу текущего приема, с точки зрения администратора или модератора – добавлять услуги клиники и распределять услуги между специалистами.
}\\
\normalsize{Web-сервис должен выполнять следующие операции:}\\
\begin{enumerate} 
\enum {1) Запись на прием. Записываться может как сам пациент, так и сам врач на приеме может его записать. Запись содержит информацию об услуге, которую нужно выбрать, врача, дату и время посещения. И в конце подтвердить запись.}\\
\enum {2) Запись на сегодня. Можно просмотреть список пациентов на сегодняшний день и начать прием.}\\
\enum3) {Текущий прием. Доступен, когда пациент пришел на прием и врач нажимает начать прием. Врач может оставлять комментарий о приеме. }\\
\enum4) {Добавление услуги. Можно добавить новую услугу, которую будет оказывать клиника. У услуги можно указать описание, стоимость и продолжительность процедуры, и сохранить введенные данные. Удобна также для просмотра описании какой-либо услуги. Услуги также можно удалять.}\\
\enum5) {Распределение услуг. Данная страница позволяет назначать определенные услуги под конкретного врача. }\\
\enum6) {История пациента. Хранится информация о предыдущих приемах данного пациента в клинике. }\\
\enum7) {Назначение лечения. }\\
\enum8) {Реализация аутентификации пользователей с помощью VK API.}\\
\end{enumerate}


\normalsize{В данной работе существуют три роли: пациента, врача и администратора. }\\


\normalsize{Роль «Пациент»: }\\
 \begin{itemize}
 \labelitemii { ознакомиться с услугами клиники;}\\
 \labelitemii { ознакомиться с лечащими врачами;}\\
 \labelitemii { записаться на лечение;}\\
 \labelitemii { просмотреть все действующие записи;}\\
 \labelitemii { отменить запись;}\\
 \labelitemii { оплатить лечение.}\\
 \end{itemize}
\normalsize{Роль «Врач»:}\\
 \begin{itemize}
 \labelitemii { посмотреть действующие записи на сегодня;}\\
 \labelitemii { начать прием;}\\
 \labelitemii { назначить лечение;}\\
 \labelitemii { завершить прием.}\\
 \end{itemize}
\normalsize{Роль «Администратор»:}\\
 \begin{itemize}
\labelitemii { подтвердить оплату пациента;}\\
\labelitemii { добавлять/удалять раздел услуг;}\\
\labelitemii { добавлять/удалять услуги;}\\
\labelitemii { распределять услуги между врачами.}\\
 \end{itemize}

\newpage

\hfill \break
\hfill \break
\begin{center}
\Large{\textbf{2. Программное обеспечение}}\\
\end{center}


\normalsize{Для реализации web-сервиса для курсовой работы был использован следующий набор программного обеспечения. 
}\\

\normalsize{•	NetBeans – свободная интегрированная среда разработки приложений (IDE) на языках программирования Java, Python, PHP, JavaScript, C, C++, Ада и ряда других. Используется для разработки серверной части.
}\\


\normalsize{•WebStorm — интегрированная среда разработки на JavaScript, CSS & HTML от компании JetBrains, разработанная на основе платформы IntelliJ IDEA. Используется для разработки клиентской части приложения. 
}\\


\normalsize{•	pgAdmin - графический клиент для работы с сервером, через который мы в удобном виде можем создавать, удалять, изменять базы данных и управлять ими.
}\\


\normalsize{• Яндекс. Облако: Yandex Managed Service for PostgreSQL для управления базами данных разного размера и использования их в разработке приложений; Yandex Container Registry - сервис для хранения, развертывания и управления Docker-образами; Yandex Compute Cloud – сервис, предоставляющий масштабируемые вычислительные мощности для размещения, тестирования и прототипирования проектов.
}\\


\normalsize{Разработка web-приложения проходила в 3 этапа: разработка БД, разработка web-сервиса, разработка клиентсткого приложения. Все этапы описаны в следующих пунктах.}\\

\newpage

\hfill \break
\hfill \break
\begin{center}
\Large{\textbf{3. Разработка базы данных
}}\\
\end{center}


\normalsize{База данных «Стоматологическая клиника» задается следующими отношениями. В скобках указаны типы данных для соответствующего атрибута.
}\\


\normalsize{«Доктор» со следующими полями:
}\\
 \begin{itemize}
 \labelitemii { ID (тип целое long);}\\
 \labelitemii { услуги (список) procedure; }\\
 \labelitemii { этапы процедур по лечению (список) visitJournal;}\\
 \labelitemii { vkId (целое);}\\
 \labelitemii { уровень админа (целое) adminLevel;}\\
 \labelitemii { статус доктора (DocumentStatus) docStatus;}\\
  \labelitemii { профессия (строка).}\\
 \end{itemize}


\normalsize{«Пациент»:
}\\
 \begin{itemize}
 \labelitemii { ID (тип целое long);
}\\
 \labelitemii { vkId (целое);}\\
 \labelitemii { запись на лечение (список) therapyJournal.}\\
 \end{itemize}
 
 
\normalsize{Услуги клиники: }\\
 \begin{itemize}
\labelitemii { ID (тип целое long);}\\
\labelitemii { доктор (список);
}\\
\labelitemii { название (строка);}\\
\labelitemii { описание (строка);
}\\
\labelitemii { цена (вещественное число);
}\\
\labelitemii { время (целое);
}\\
\labelitemii {этапы процедур по лечению (список).
}\\
 \end{itemize}


\normalsize{Журнал терапии: }\\
 \begin{itemize}
\labelitemii { ID (тип целое long);}\\
\labelitemii { пациент;
}\\
\labelitemii { этапы процедур по лечению (список);
}\\
\labelitemii { дата (java.sql.Date);
}\\
\labelitemii { время (целое);
}\\
\labelitemii { статус (JournalStatus).
}\\
 \end{itemize}
 
 
 \normalsize{Журнал элементов лечения: }\\
 \begin{itemize}
\labelitemii { ID (тип целое long);}\\
\labelitemii { доктор;
}\\
\labelitemii { запись на лечение;
}\\
\labelitemii {процедура;
}\\
\labelitemii { дата;
}\\
\labelitemii { время (целое);
}\\
\labelitemii {статус;
}\\
\labelitemii { комментарий (строка);
}\\
\labelitemii { номер (целое);
}\\
\labelitemii { notLater (целое).
}\\
 \end{itemize}
 
 
 \normalsize{«Долги»:
}\\
 \begin{itemize}
 \labelitemii { сумма;
}\\
 \labelitemii { статус;}\\
 \labelitemii { дата.}\\
 \end{itemize}
 
 
  \normalsize{«Поступление денежных средств»:
}\\
 \begin{itemize}
 \labelitemii { стоимость;
}\\
 \labelitemii { время;}\\
 \labelitemii { дата.}\\
 \labelitemii { тип перевода;}\\
 \labelitemii { ФИО администратора.}\\
 \end{itemize}
 \newpage

\hfill \break
\hfill \break
\begin{center}
\Large{\textbf{4. Разработка web-сервиса
}}\\
\end{center}


\normalsize{Для реализации web-сервиса использовалась технология REST. Работа велась в интегрированная среде разработки программного обеспечения NetBeans.
}\\


\normalsize{Реализация пользовательской аутентификации была сделана с помощью VK API. API ВКонтакте — это интерфейс, который позволяет получать информацию из базы данных vk.com с помощью http-запросов к специальному серверу. Не нужно знать в подробностях, как устроена база, из каких таблиц и полей каких типов она состоит — достаточно того, что API-запрос об этом «знает». Синтаксис запросов и тип возвращаемых ими данных строго определены на стороне самого сервиса. 
}\\


\normalsize{Так как вк приложение использует пользовательскую аутентификацию, сбор и передачу персональных данных, то требуется безопасность конфидициальных данных. Поэтому требуется использование безопасного протокола https. Для реализации передачи данных посредством HTTPS на веб-сервере, обрабатывающем запросы от клиентов, должен быть установлен специальный SSL-сертификат.
}\\


\normalsize{На сайте https://www.reg.ru/ был куплен уникальный домен https://javarsatu.site. Вместе с ним был выдан SSL-сертификат со сроком действия 1 год. DV (domain validation) SSL — подтверждает домен, а также шифрует и защищает данные при передаче с помощью протокола https. 
SSL-сертификат состоит из двух ключей, с помощью которых информация кодируется — публичного и секретного. Публичный ключ известен всем. Он шифрует информацию при передаче. Секретный ключ расшифровывает информацию на сервере и известен только владельцу домена.
}\\


\normalsize{Браузер проверяет подлинность сертификата перед установкой соединения с сайтом. Проверка прошла успешно —в адресной строке зеленый замочек и надпись “надёжный”.
}\\


\normalsize{Тип приложения - IFrame/Flash приложение. Они загружаются непосредственно на сервер ВКонтакте (Flash) или встраиваются во фрейме с внешнего сайта.
}\\


\normalsize{Для идентификации в API используется специальный ключ доступа, который называется access\_token. Токен — это строка из цифр и латинских букв, которая передается на сервер вместе с запросом. Из этой строки сервер получает всю нужную ему информацию.
}\\


\normalsize{Тип http – запроса Post. Был выбран этот тип запроса по причине: передает данные таким образом, что пользователь сайта уже не видит передаваемые скрипту данные, тем самым лучший способ обезопасить передаваемые данные; позволяет передавать запросу файлы; а также передает больший объем информации.
}\\


\normalsize{Были созданы 5 Entity-классов (классы-сущности), которые служат для привязки к сущностям из базы данных. Они содержат описание полей каждой сущности, а также геттеры и сеттеры для них. Для связи полей из Entity-класса и полей из БД использовались следующие аннотации:
}\\


\normalsize{1)	@Entity – указывает, что данный класс является сущностью (в скобках у данной аннотации указывается имя сущности);}\\


\normalsize{2)	@Id – обязательное поле в каждой сущности, указывает, что поле является первичным ключом;}\\


\normalsize{3)	@Column("название\_поля") – определяет к какому столбцу в таблице БД относится конкретное поле класса (аттрибут класса). }\\
\normalsize {Наиболее часто используемые аттрибуты аннотации @Column такие:
•	name - указывает имя столбца в таблице
•	unique - определяет, должно ли быть данное значение уникальным
•	nullable - определяет, может ли данное поле быть NULL, или нет
•	Length - указывает, какой размер столбца}\\
\normalsize {Данную аннотацию можно использовать без указания названия поля, тогда придётся называть поле в Entity-классе точно также как поле в базе данных;}\\


\normalsize{4)	@GeneratedValue - значение первичного ключа генерируется автоматически. Может использоваться 4 типа генерации: AUTO, IDENTITY, SEQUENCE, TABLE. Если не указано значение явно, типом генерации по умолчанию является AUTO. Если мы используем тип генерации по умолчанию, поставщик сохраняемости будет определять значения на основе типа атрибута первичного ключа. Этот тип может быть числовым или UUID. Для числовых значений генерация основана на генераторе последовательности или таблицы, в то время как значения UUID будут использовать UUIDGenerator;}\\


\normalsize{5)	@JsonbTransient - игнорировать поле во время десериализации / сериализации;}\\


\normalsize{6)	@Enumerated - для облегчения использования перечислений JPA предоставляет данную аннотацию, которая указывает, что свойство должно обрабатываться как перечисление. Значение EnumType указывает в какой форме перечисления должны быть сохранены в базе данных. EnumType.STRING указывает на то, что значение перечисления должно быть сохранено строкой. При создании таблицы с EnumType.STRING будет создано VARCHAR поле;}\\


\normalsize {7)	@OneToOne - При связи один-к-одному каждая запись в одной таблице напрямую связана с отдельной записью в другой таблице;}\\


\normalsize{8)	@ManyToOne, @OneToMany – данные аннотации устанавливают связь дочернего класса и родительского. Для того, чтобы объявить сторону, которая не несет ответственности за отношения, используется атрибут mappedBy. Каскадирование позволяет указать JPA, что необходимо «сделать со связанным объектом при выполнении операции с владельцем»;}\\


\normalsize {9)	@ManyToMany - определяет отношение многие ко многим. Когда объект Entity класса с одной стороны содержит коллекцию объекты Entity классов другой стороны. В свою очередь объекты Entity классов со второй стороны могут содержать коллекцию объектов Entity классов первой стороны.}\\


\normalsize {Далее были разработаны REST-классы, с помощью которых ведётся работа WEB-приложения. Полностью описывать классы нет смысла поэтому далее будут рассмотрены основные моменты. 
}\\


\normalsize {Можно увидеть в rest/api следующие файлы обработчиков запросов:  DataSource - выдает данные из базы; DataReceiver - принимает данные для записи в базу; DataDelete - удаляет данные из базы.
}\\


\normalsize {Каждый компонент помечается аннотацией @Path(“путь”). Эта аннотация включает атрибут, задающий путь к ресурсу. Его значением могут быть строка символов, переменные, а также переменные в сочетании с регулярным выражением. }\\


\normalsize {Далее создавались методы в каждом классе, которые принимают запросы от пользователя, в соответствии с ними берут данные из базы данных и отдают их запросившему. Каждый метод помечался аннотациями: @POST (отвечает за тип HTTP запроса релевантный этому методу),
@Produces (содержит тип, который возвращает метод), 
@Consumes (содержит тип, который принимает метод), 
@Path (указывает путь, по которому можно обратиться к данному методу),
@Transactional (указывает, что метод должен выполняться в транзакции).
 }\\


\normalsize {Общение между клиентом и сервером происходит с помощь JSON-запросов (ответов). Для передачи данных от WEB-сервиса не нужно их вручную упаковывать объекты в формат JSON, это происходит автоматически. А при приёме сообщения в формате JSON, его нужно анализировать либо вручную, либо с помощью специальных библиотек.}\\

\newpage

\hfill \break
\hfill \break
\begin{center}
\Large{\textbf{5. Разработка клиентского приложения
}}\\
\end{center}


\normalsize{Следующим шагом была разработка клиентского приложения. Для этого использовался фреймворк Angular. Работа велась в среде разработки WebStorm.
}\\


\normalsize{Были созданы компоненты, отвечающие за отображение интерфейса для работы с сущностями.
}\\


\normalsize{Сам интерфейс можно поделить на три части:
}\\


\normalsize{1.	Меню навигации - отвечает за переключение между страницами. При нажатии на соответствующий раздел, открывается компонент для работы с ним.
}\\


\normalsize{1.	Кнопки или параметры, при помощи которых происходит взаимодействие с таблицей. 
}\\


\normalsize{1.	Формы, в которых отображается информация из базы данных. 
}\\


\normalsize{Service — слой данных в приложении, отвечающий за выполнение бизнес-логики. Если программа должна выполнить какую-то бизнес-логику — она делает это через сервисы. 
}\\


\newpage

\hfill \break
\hfill \break
\begin{center}
\Large{\textbf{Заключение

}}\\
\end{center}


\normalsize{В ходе выполнения курсовой работы было написано WEB-приложение «Стоматологическая клиника». Разработка проходила в три этапа: разработка и создание базы данных, написание WEB-сервиса, написание клиентского приложения.

}\\


\normalsize{Проверка базы данных проводилось с помощью запросов к ней, для удобства использовался менеджер баз данных pgAdmin. Для тестирования клиентского приложения использовался браузер Google Chrome.

}\\


\normalsize{В результате работы был опробован архитектурный стиль разработки  – REST. Так же была изучена разработка web интерфейсов при помощи Angular и Quarkus. Так же были изучены особенности работы с Яндекс.Облако, а именно работа с виртуальной машиной, работа с докер-контейнерами, работа с БД.

}\\


\normalsize{И итого, после тестирования, можно с уверенностью сказать, что поставленные задачи выполнены и цель данной курсовой работы достигнута.
}\\


\newpage

\hfill \break
\hfill \break
\begin{center}
\Large{\textbf{Список использованных источников
}}\\
\end{center}


\normalsize{1.	Таблицы в Angular Material. [Электронный ресурс]. – Электрон. текстовые, граф. дан. Режим доступа: https://toster.ru/q/530230
}\\


\normalsize{2. pgAdmin [Электронный ресурс] URL:https:// habr.com/ pgdayrussia/.
}\\


\normalsize{3.	Representational State Transfer. [Электронный ресурс]. – Электрон. текстовые, граф. дан. Режим доступа: https:// en.wikipedia.org/ wiki/ Representational \_state \_transfer
}\\


\normalsize{4.	Работа с формами. [Электронный ресурс]. – Электрон. текстовые, граф. дан. Режим доступа: https://metanit.com/web/angular2/5.1.php
}\\


\normalsize{5.	Angular Material. [Электронный ресурс]. – Электрон. текстовые, граф. дан. Режим доступа: https://material.angular.io/components/categories (дата обращения: 25.02.2019)

}\\

\end{document}  % КОНЕЦ ДОКУМЕНТА !
